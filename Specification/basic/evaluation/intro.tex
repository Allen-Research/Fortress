%%%%%%%%%%%%%%%%%%%%%%%%%%%%%%%%%%%%%%%%%%%%%%%%%%%%%%%%%%%%%%%%%%%%%%%%%%%%%%%%
%   Copyright 2009, Oracle and/or its affiliates.
%   All rights reserved.
%
%
%   Use is subject to license terms.
%
%   This distribution may include materials developed by third parties.
%
%%%%%%%%%%%%%%%%%%%%%%%%%%%%%%%%%%%%%%%%%%%%%%%%%%%%%%%%%%%%%%%%%%%%%%%%%%%%%%%%

The state of an executing Fortress program consists of a set of
\emph{threads}, and a \emph{memory}.  Communication with the outside
world is accomplished through \emph{input and output actions}.  In
this chapter, we give a general overview of the execution process and
introduce terminology used throughout this specification
to describe the behavior of Fortress constructs.

Executing a Fortress program consists of \emph{evaluating}
the body expression of the \VAR{run} function and
the initial-value expressions of all top-level variables and singleton
object fields in parallel.
All initializations must complete normally before any reference to the
objects or variables being initialized.

Threads evaluate expressions by taking \emph{steps}.
We say evaluation of an expression \emph{begins} when the first step is
taken.
A step may \emph{complete} the evaluation, in which case no more steps
are possible on that expression, or it may result in an
\emph{intermediate expression}, which requires further evaluation.
\emph{Dynamic program order} is a partial order among the expressions
evaluated in a particular execution of a program.  When two
expressions are ordered by dynamic program order, the first must
complete execution before any step can be taken in the second.
\chapref{expressions} gives the dynamic program order constraints for
each expression in the language.

Intermediate expressions
are generalizations of ordinary Fortress expressions:
some intermediate expressions cannot be written in programs.
We say that one expression
is \emph{dynamically contained} within a second expression
if all steps taken in evaluating the first expression
must be taken between the beginning and completion of the second.
A step may also have effects on the program state
beyond the thread taking the step,
for example, by modifying one or more locations in memory,
creating new threads to evaluate other expressions,
or performing an input or output action.
Threads are discussed further in \secref{threads-parallelism}.
The memory consists of a set of \emph{locations},
which can be \emph{read} and \emph{written}.
New locations may also be \emph{allocated}.
The memory is discussed further in \secref{memory-ops}.
Finally, \emph{input actions} and \emph{output actions} are described in \secref{io-actions}.
